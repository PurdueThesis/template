\ProvidesFile{ap-tables.tex}[2022-10-05 tables appendix]

\begin{VerbatimOut}{z.out}
\chapter{TABLES}
\ix{table}
\end{VerbatimOut}

\MyIO


% \newlength{\ta}
% \newlength{\tb}
% \newlength{\tc}
% 
% \settowidth{\ta}{\vbox{\hbox{Money}\hbox{Market}}}
% \settowidth{\tb}{\vbox{\hbox{Stocks}\hbox{and}\hbox{Bonds}}}
% \settowidth{\tc}{\vbox{\hbox{Money}\hbox{Market}\hbox{and}\hbox{Stocks}}}
% 
% {
%   \renewcommand{\baselinestretch}{1}
%   \begin{table}
%     \caption{%
%       \hfil Allocation of the IRA and Keogh Wealth\hfil\break
%       \mbox{}\hfil for Investors With or Without Brokerage Accounts\hfil
%     }
%     \label{tab:ira}
%     \begin{center}
%       \begin{tabular}%
%         {%
%           |%
%           c%
%           |%
%           >{\centering\hspace{0pt}}m{\the\ta}%  Money Market
%           |%
%           c%                                    Stocks 
%           |%
%           c%                                    Bonds
%           |%
%           c%                                    Diversified
%           |%
%           >{\centering\hspace{0pt}}m{\the\tb}%  Stocks and Bonds
%           |%
%           >{\centering\hspace{0pt}}m{\the\tc}%  Money Market and Stocks
%           |%
%           c%                                    Others
%           |%
%         }
%         \hline
%         IMP&
%           Money Market&
%           Stocks&
%           Bonds&
%           Diversified&
%           Stocks and Bonds&
%           Money Market and Stocks&
%           Others\tabularnewline
%         \hline
%         1& 14.19\%& 57.71\%& 12.21\%& 4.50\%& 7.36\%& 3.04\%& 0.99\%\tabularnewline \hline
%         2& 14.08\%& 58.18\%& 12.32\%& 4.44\%& 7.30\%& 2.80\%& 0.88\%\tabularnewline \hline
%         3 &14.26\%& 58.09\%& 12.27\%& 4.50\%& 7.19\%& 2.75\%& 0.94\%\tabularnewline \hline
%         4 &13.94\%& 58.11\%& 12.14\%& 4.78\%& 7.35\%& 2.68\%& 0.99\%\tabularnewline \hline
%         5 &13.92\%& 58.13\%& 11.93\%& 4.56\%& 7.60\%& 2.98\%& 0.88\%\tabularnewline \hline
%       \end{tabular}
%     \end{center}
%     This table presents the allocations of the wealth in the IRA
%     and Keogh accounts in various asset classes.
%     Results from each set of imputed data are presented here.
%     The first column lists the number of the imputations,
%     and rest of the columns lists various allocations.
%     Entrees under each asset class show the percentage of investors
%     who have most of their IRA
%     and Keogh wealth invested in that particular asset class.
%     The asset class Diversified
%     includes stocks,
%     bonds,
%     and money market investments.
%     The asset class Others
%     include investments in various life insurance products,
%     annuities,
%     real estate, etc.
%     \medskip
%     \footnotesize SOURCE: Survey of Consumer Finances,
%     2001,
%     Federal Reserve Board,
%     USA.\par
%   \end{table}
% }


\begin{VerbatimOut}{z.out}

Here is a really simple table.
I was greatly influenced
by Herbert Voss' following ideas
on typsetting tables
\cite{voss2011}:
Use |\toprule|, |\midrule|, and |\bottomrule|.
\index{\verb+\toprule+}
\index{\verb+\midrule+}
\index{\verb+\bottomrule+}
Don't have blank horizontal space to the left
or right of body of table.
\ix{Voss, Herbert}

% "h" means put table "here"---don't let it float to top or bottom of page
\begin{table}[ht]
  \caption{The first three American Presidents.}
  \vspace*{6pt}
  \centering
    % Table format:
    %     WHAT    DESCRIPTION
    %     @{}     don't put extra space before first column
    %     r       right justify first column
    %     l       left justify second column
    %     @{}     don't put extra space after second column
    \begin{tabular}{@{}rl@{}}
      \toprule
      \bf Number& \bf Name\\
      \midrule
      1& George Washington\\
      2& John Adams\\
      3& Thomas Jefferson\\
      \bottomrule
    \end{tabular}
  \label{ta:first-three-american-presidents}
\end{table}
\ix{table}
\index{\verb+\begin{table}+}
\end{VerbatimOut}

\MyIO


\begin{VerbatimOut}{z.out}

\newpage

Here is the same table with a longer caption.

% "h" means put table "here"---don't let it float to top or bottom of page
\begin{table}[ht]
  \caption{%
    The first three American Presidents.
    This caption is
    much, much, much, much, much, much,
    much, much, much, much, much, much
    longer.%
  }
  \vspace*{6pt}
  \centering
    % Table format:
    %     WHAT    DESCRIPTION
    %     @{}     don't put extra space before first column
    %     r       right justify first column
    %     l       left justify second column
    %     @{}     don't put extra space after second column
    \begin{tabular}{@{}rl@{}}
      \toprule
      \bf Number& \bf Name\\
      \midrule
      1& George Washington\\
      2& John Adams\\
      3& Thomas Jefferson\\
      \bottomrule
    \end{tabular}
  \label{ta:first-three-american-presidents-longer-caption}
\end{table}
\end{VerbatimOut}

\MyIO


\begin{VerbatimOut}{z.out}

\newpage

\LaTeX\ can print horizontal
and vertical rules in tables.
I don't like the way this looks 
and suggest you do not use tables
with lots of horizontal and vertical lines.
\begin{table}[ht]
  \caption{The first three American Presidents with horizontal and vertical lines}
  \vspace*{6pt}
  \centering
    % Table format:
    %     WHAT    DESCRIPTION
    %     @{}     don't put any space left of first column
    %     |       print a vertical rule
    %     c       center column 
    %     |       print a vertical rule
    %     l       left justify column
    %     |       print a vertical rule
    %     @{}     don't put any space right of last column
    \begin{tabular}{@{}|c|l|@{}}
      % "\hline" prints a horizontal rule
      \hline
      \bf \#& \bf Name\\
      \hline
      1& George Washington\\
      \hline
      2& John Adams\\
      \hline
      3& Thomas Jefferson\\
      \hline
    \end{tabular}
  \label{ta:American-Presidents-with-horizontal}
\end{table}
\end{VerbatimOut}

\MyIO


\begin{VerbatimOut}{z.out}

\newpage

Here is a more complicated table.

{
  \UndefineShortVerb{\|}
\begin{table}[ht]
  \caption{C Bitwise Operators}
  \vspace*{6pt}
  \centering
    % Table format:
    %     WHAT    DESCRIPTION
    %     @{}     don't put extra space before first column
    %     c       first column is centered
    %     c       second column is centered
    %     c       third column is centered
    %     c       fourth column is centered
    %     @{}     don't put extra space after fourth column
    \begin{tabular}{@{}cccc@{}}
      \toprule
      \bf A& \bf B& \bf A\(|\)B& \bf A\&B\\[2pt]
      \midrule
      0& 0& 0& 0\\
      0& 1& 1& 0\\
      1& 0& 1& 0\\
      1& 1& 1& 1\\
      \bottomrule
    \end{tabular}
  \label{ta:C-Bitwise}
\end{table}
}
\end{VerbatimOut}

\MyIO


% Plain Tex's \halign command can be used to make tables but it is not
% worth telling users about.  LaTeX is more convenient to make tables
% with generally.
% 
% \begin{VerbatimOut}{z.out}
% 
% You can use Plain \TeX's \verb+\halign+ command to make tables also.
% If you can't do a complicated table using \LaTeX\ commands
% you may want to try using Plain \TeX\ commands.
% \LaTeX's table making commands use Plain \TeX\ commands.
% 
% \begin{table}[ht]
%   \caption{American Presidents using {\tt\char'134 halign}}
%   \hbox to \textwidth{\hss\vbox{\halign{%
%     \strut #&      % 0. \strut
%     \hfil#\qquad&  % 1. Number
%     #\hfil\cr      % 2. Name
%     %
%     & \bf Number& \bf Name\cr
%     & 1& George Washington\cr
%     & 2& John Adams\cr
%     & 3& Thomas Jefferson\cr
%   }}\hss}
%   \label{ta:American-Presidents-using}
% \end{table}
% \end{VerbatimOut}
% 
% \MyIO


\begin{VerbatimOut}{z.out}
\begin{table}[ht]
  \caption{Participant descriptors for twelve practitioners engaged in co-creation activities}
  \label{tab:22participants}
  \center
  \begin{tabular}{@{}cllS@{}}
    \toprule
    \multicolumn{1}{@{}l}{\textbf{Pseudonym}}&
      \textbf{Disciplinary Role}&
      \textbf{Company Type}&
      \multicolumn{1}{l@{}}{\textbf{\# Years of Experience}}\\
    \midrule
    \multicolumn{4}{@{}l@{}}%
    {%
      \textbf{Sequence 1:} $\text{A1.1}\to\text{B2.1}$:
      Overlapping dilemma cards to strengthen and represent%
    }\\
    \multicolumn{4}{@{}l}{ethical complexity
      through practitioner's current ecological complexity model}\\
    1P1& UX Designer& Enterprise (B2C)& 1.5\\
    1P2& Product Manager& Enterprise (B2B)& 5\\
    1P3& Data Scientist& Agency or Consultancy& 1\\
    \noalign{\vspace{8pt}}
    \multicolumn{4}{@{}l@{}}%
    {%
      \textbf{Sequence 2:} $\text{B2.1}\to\text{A1.1}$:
      Building and tracing complexity based on Dilemmas Cards%
    }\\
    \multicolumn{4}{@{}l}{to reconstruct and reflect on their experience}\\
    2P1& UX Designer& Agency or Consultancy& 8\\
    2P2& Product Manager& Agency or Consultancy& 2\\
    2P3& Software Engineer& Enterprise (B2B)& 2\\
    \bottomrule
  \end{tabular}
\end{table}
\end{VerbatimOut}

\MyIO


\begin{VerbatimOut}{z.out}

\newpage

Here is a table that is too long to fit on one page.

% This is very loosely based on page 106 of _A Guide to LaTeX_, third edition,
% by Helmut Kopka and Patrick W. Daly.
\begin{longtable}{@{}ll@{}}
    \caption{State Abbreviations}\\
    \toprule
    \bf State& \bf Abbreviation\\
    \hline
  \endfirsthead
    \caption[]{\emph{continued}}\\
    \midrule
    \bf State& \bf Abbreviation\\
    \midrule
  \endhead
    \hline
    \multicolumn{2}{r}{\emph{continued on next page}}
  \endfoot
    \bottomrule
  \endlastfoot
  Alabama& AL\\
  Alaska& AK\\
  American Samoa& AS\\
  Arizona& AZ\\
  Arkansas& AR\\
  Armed Forces Europe& AE\\
  Armed Forces Pacific& AP\\
  Armed Forces the Americas& AA\\
  California& CA\\
  Colorado& CO\\
  Connecticut& CT\\
  Delaware& DE\\
  District of Columbia& DC\\
  Federated States of Micronesia& FM\\
  Florida& FL\\
  Georgia& GA\\
  Guam& GU\\
  Hawaii& HI\\
  Idaho& ID\\
  Illinois& IL\\
  Indiana& IN\\
  Iowa& IA\\
  Kansas& KS\\
  Kentucky& KY\\
  Louisiana& LA\\
  Maine& ME\\
  Marshall Islands& MH\\
  Maryland& MD\\
  Massachusetts& MA\\
  Michigan& MI\\
  Minnesota& MN\\
  Mississippi& MS\\
  Missouri& MO\\
  Montana& MT\\
  Nebraska& NE\\
  Nevada& NV\\
  New Hampshire& NH\\
  New Jersey& NJ\\
  New Mexico& NM\\
  New York& NY\\
  North Carolina& NC\\
  North Dakota& ND\\
  Northern Mariana Islands& MP\\
  Ohio& OH\\
  Oklahoma& OK\\
  Oregon& OR\\
  Pennsylvania& PA\\
  Puerto Rico& PR\\
  Rhode Island& RI\\
  South Carolina& SC\\
  South Dakota& SD\\
  Tennessee& TN\\
  Texas& TX\\
  Utah& UT\\
  Vermont& VT\\
  Virgin Islands& VI\\
  Virginia& VA\\
  Washington& WA\\
  West Virginia& WV\\
  Wisconsin& WI\\
  Wyoming& WY\\
  \multicolumn{2}{c}{make this three pages long}\\
  \multicolumn{2}{c}{make this three pages long}\\
  \multicolumn{2}{c}{make this three pages long}\\
  \multicolumn{2}{c}{make this three pages long}\\
  \multicolumn{2}{c}{make this three pages long}\\
  \multicolumn{2}{c}{make this three pages long}\\
  \multicolumn{2}{c}{make this three pages long}\\
  \multicolumn{2}{c}{make this three pages long}\\
  \multicolumn{2}{c}{make this three pages long}\\
  \multicolumn{2}{c}{make this three pages long}\\
  \multicolumn{2}{c}{make this three pages long}\\
  \multicolumn{2}{c}{make this three pages long}\\
  \multicolumn{2}{c}{make this three pages long}\\
  \multicolumn{2}{c}{make this three pages long}\\
  \multicolumn{2}{c}{make this three pages long}\\
  \multicolumn{2}{c}{make this three pages long}\\
  \multicolumn{2}{c}{make this three pages long}\\
  \multicolumn{2}{c}{make this three pages long}\\
  \multicolumn{2}{c}{make this three pages long}\\
  \multicolumn{2}{c}{make this three pages long}\\
  \multicolumn{2}{c}{make this three pages long}\\
  \multicolumn{2}{c}{make this three pages long}\\
  \multicolumn{2}{c}{make this three pages long}\\
  \multicolumn{2}{c}{make this three pages long}\\
  \multicolumn{2}{c}{make this three pages long}\\
  \multicolumn{2}{c}{make this three pages long}\\
  \multicolumn{2}{c}{make this three pages long}\\
  \multicolumn{2}{c}{make this three pages long}\\
  \multicolumn{2}{c}{make this three pages long}\\
  \multicolumn{2}{c}{make this three pages long}\\
  \multicolumn{2}{c}{make this three pages long}\\
  \multicolumn{2}{c}{make this three pages long}\\
  \multicolumn{2}{c}{make this three pages long}\\
  \multicolumn{2}{c}{make this three pages long}\\
  \multicolumn{2}{c}{make this three pages long}\\
\end{longtable}
\end{VerbatimOut}

\MyIO


\begin{VerbatimOut}{z.out}

% The table is on the next page.

\newpage

% Set \LTcapwidth (the longtable caption width)
% to \textwidth minus 4 paragraph indent widths.
\setlength{\LTcapwidth}{\textwidth}
\addtolength{\LTcapwidth}{-4\parindent}

\newlength{\twidth}
\newlength{\theight}

\setlength{\twidth}{\textwidth}
\setlength{\theight}{\textheight}

\begin{sidewaystable}
  % The following two lines compensate for what I think is a bug.
  \setlength{\textwidth}{\theight}
  \setlength{\textheight}{\twidth}
  \caption{Sidewaystable of the first three American Presidents.}
  \vspace*{6pt}
  \centering
    \begin{tabular}{@{}rl@{}}
      \toprule
      \bf Number& \bf Name\\
      \midrule
      1& George Washington\\
      2& John Adams\\
      3& Thomas Jefferson\\
      \bottomrule
    \end{tabular}
\end{sidewaystable}
\end{VerbatimOut}

\MyIO

\begin{VerbatimOut}{z.out}
\begin{sidewaystable}
  % The following two lines compensate for what I think is a bug.
  \setlength{\textwidth}{\theight}
  \setlength{\textheight}{\twidth}
  \caption{Two tables can be placed vertically in a sidewaystable environment.}
  \vspace*{6pt}
  \centering
    \begin{tabular}{@{}rl@{}}
      \toprule
      \bf Number& \bf Name\\
      \midrule
      1& George Washington\\
      2& John Adams\\
      3& Thomas Jefferson\\
      \bottomrule
    \end{tabular}
  \vspace*{2\baselineskip}
  \caption{This is the second table in the sideways environment.}
  \vspace*{6pt}
    \begin{tabular}{@{}rl@{}}
      \toprule
      \bf Number& \bf Name\\
      \midrule
      1& George Washington\\
      2& John Adams\\
      3& Thomas Jefferson\\
      \bottomrule
    \end{tabular}
\end{sidewaystable}
\end{VerbatimOut}

\MyIO


% Plain Tex's \halign command can be used to make tables but it is not
% worth telling users about.  LaTeX is more convenient to make tables
% with generally.
% 
% \begin{VerbatimOut}{z.out}
%
% \begin{sidewaystable}
%   % The following two lines compensate for what I think is a bug.
%   \setlength{\textwidth}{\theight}
%   \setlength{\textheight}{\twidth}
%   \caption{%
%     sidewaystable
%     {\tt\cbackslash halign\copencurly}\ldots{\tt\cclosecurly\/} table%
%   }
%   \hbox to \textwidth{\hss\vbox{\halign{%
%     \strut #&      % 0. \strut
%     \hfil#\qquad&  % 1. Number
%     #\hfil\cr      % 2. Name
%     %
%     & \bf Number& \bf Name\cr
%     \noalign{\vskip 2pt}
%     & 1& George Washington\cr
%     & 2& John Adams\cr
%     & 3& Thomas Jefferson\cr
%   }}\hss}
% \end{sidewaystable}
% \end{VerbatimOut}
%
% \MyIO


\begin{VerbatimOut}{z.out}
\begin{sidewaystable}[ht]%
  % The following two lines compensate for what I think is a bug.
  \setlength{\textwidth}{\theight}%
  \setlength{\textheight}{\twidth}%
  \caption{Live Guitar Open String Testing Data - Pitch (\textit{f\textsubscript{0}})}
  \vspace*{6pt}%
  \label{ta:live-guitar}%
  % Define "Live Guitar Test" column.
  \def\lgt#1{\bf Live Guitar Test #1}
  % Define "Note", "Computed", "Measured", "%", and "Accuracy" column headings.
  \def\note{\bf Note}
  \def\cal{\bf Computed}
  \def\mea{\bf Measured}
  \def\per{\bf \%}
  \def\acc{\bf Accuracy}
  % Define "Name", "f_0 (Hz)", "Error", and "Range (\textcent)" column headings.
  \def\name{\bf Name}
  \def\fsz{\bf \textit{f\textsubscript{0}} (Hz)}
  \def\err{\bf Error}
  \def\ran{\bf Range (\textcent)}
  % Make "!" be an invisible character the width of a digit.
  % (All digits in the normal font are the same width.)
  \catcode`\!=\active    \def!{\hphantom 1}
  \hbox to \textwidth
  {%
    \hss
    % From http://zerocapcable.com/?page_id=225
    %     The units of tuning accuracy are cents. A cent is one hundredth
    %     of a semitone.  Since there are 12 semitones in an octave, there
    %     are 1200 cents in an octave.
    % The default \tabcolsep is 6.0pt.
    \setlength{\tabcolsep}{5pt}%
    \begin{tabular}{@{}cc|ccc|ccc|ccc@{}}
      \hline
      \multicolumn{2}{c|}{ }&
        \multicolumn{3}{c|}{\lgt1}&
        \multicolumn{3}{c|}{\lgt2}&
        \multicolumn{3}{c}{\lgt3}\\
      \cline{3-11}
      \note& \cal& \mea& \per& \acc& \mea& \per& \acc& \mea& \per& \acc\\
      \name& \fsz& \fsz& \err& \ran& \fsz& \err& \ran& \fsz& \err& \ran\\
      \hline
      E\textsubscript 2& !82.407& !82.333& 0.0897& $+2$& !82.616& 0.2538& $+6$& !82.474& 0.0814& $+2$\\
      A\textsubscript 2& 110.000& 110.092& 0.0836& $+2$& 110.092& 0.0836& $+2$& 110.092& 0.0836& $+2$\\
      D\textsubscript 3& 146.832& 146.789& 0.0295& $-2$& 146.789& 0.0295& $-2$& 147.239& 0.2769& $+6$\\
      G\textsubscript 3& 195.998& 196.721& 0.3690& $+8$& 195.918& 0.0407& $+2$& 196.721& 0.3690& $+8$\\
      B\textsubscript 3& 246.942& 247.423& 0.1949& $+4$& 246.517& 0.1720& $-4$& 247.423& 0.1949& $+4$\\
      E\textsubscript 4& 329.628& 331.034& 0.4267& $+8$& 331.034& 0.4267& $+8$& 331.034& 0.4267& $+8$\\
      \hline
      \multicolumn{11}{@{}l}{Thanks to Kathryn Schmidt for donating this table.}\\
    \end{tabular}
    \hss
  }
\end{sidewaystable}
\end{VerbatimOut}

\MyIO


\begin{VerbatimOut}{z.out}
% Define a control sequence to save typing.
% Let * represent zero or more spaces!
% Method 1: \def\g#1{ requires using \g*{10} for 10.
%           Two shifted characters, { and } are needed.
% Method 2: \def\g#1/{ requires using \g*10/ for 10.
%           One unshifted character, / is needed.
% Method 2 requires less work than Method 1.
\def\g#1/{\includegraphics[scale=0.5]{gr-metapost-tally-#1.pdf}}%

% Define a length for use later.
\newlength{\tlen}
\setlength{\tlen}{2\parindent}
\end{VerbatimOut}

\MyIO


\begin{VerbatimOut}{z.out}
\begin{table}[h]%
  \label{ta:first-tally-table}
  \caption
  [%
    First tally table.  Use this method.%
  ]%
  {%
    First tally table.  Use this method.  I think it is the simplest.
  }
  \vspace*{6pt}
  %   Note that tabular* instead of tabular is used below.
  %   The {\textwidth} makes the total width of the table the width
  % of the printed area of the page.
  %   The @{\kern\tlen} puts blank space the width of two paragraph indents
  % before the first column.
  %   The @{extracolsep{\fill}} adds \fill space between all subsequent
  % columns.
  %   The lll left justifies the next three columns.
  % after the column.
  %   The @{\kern\tlen} puts blank space the width of two
  % paragraph indents before the first column.
  \begin{tabular*}{\textwidth}{@{\kern\tlen}@{\extracolsep{\fill}}lll@{\kern\tlen}}%
    \g 01/& \g 02/& \g 03/\\
    \g 04/& \g 05/\\
  \end{tabular*}%
\end{table}
\end{VerbatimOut}

\MyIO


\begin{VerbatimOut}{z.out}
\begin{table}[h]
  \caption{%
    Second tally table.
    Don't use this method.
    The method used in the first tally table
    is easier to understand.%
  }%  
  \vspace*{6pt}
  %   Note that tabularx instead of tabular is used below.
  %   The {\textwidth} makes the total width of the table the width
  % of the printed area on the page.
  %   The @{\kern\tlen} puts blank space the width of two paragraph indents
  % before the first column.
  %   The XX makes the first two columns the same width including the space
  % after the column.
  %   The l left justifies the last column.
  %   The @{\kern\tlen} puts blank space the width of two paragraph indents
  % after the last column.
  \begin{tabularx}{\textwidth}{@{\kern\tlen}XXl@{\kern\tlen}}%
    \g 01/& \g 02/& \g 03/\\
    \g 04/& \g 05/\\
  \end{tabularx}%
\end{table}
\end{VerbatimOut}

\MyIO


\begin{VerbatimOut}{z.out}
\begin{table}[h!]
  \caption{
    Third tally table.
    Don't use this method.
    The method used in the first tally table
    is easier to understand.%
  }%
  \vspace*{6pt}
  \def\t #1/#2/#3/%
  {%
    \hbox to\textwidth{%
      \kern\tlen \g #1/\hfil \g #2/\hfil \g #3/\kern\tlen
    }%
  }%
  \vbox{
    \t 01/02/03/
    \hbox to\textwidth{%
      \kern\tlen \g 04/\hfil \g 05/\hfil \phantom{\g 05/}\kern\tlen
    }%
    }
  \end{table}
\end{VerbatimOut}

\MyIO
  

\begin{VerbatimOut}{z.out}


% Process all unprocessed floats.
% None of the current floats will be after the \FloatBarrier.
\FloatBarrier
\end{VerbatimOut}

\MyIO





%\newlength{\ta}
%\settowidth{\ta}{\vbox{\hbox{Money}\hbox{Market}}}
%\newlength{\tb}
%\settowidth{\tb}{\vbox{\hbox{Stocks}\hbox{and}\hbox{Bonds}}}
%\newlength{\tc}
%\settowidth{\tc}{\vbox{\hbox{Money}\hbox{Market}\hbox{and}\hbox{Stocks}}}
%
%  {\renewcommand{\baselinestretch}{1}
%\begin{table}
%  \caption{\hfil Allocation of the IRA and Keogh Wealth\hfil\break\mbox{}\hfil for Investors With or Without Brokerage Accounts\hfil}
%  \label{tab:ira}
%  \begin{center}
%    \begin{tabular}%
%      {%
%        |%
%        c%
%        |%
%        >{\centering\hspace{0pt}}m{\the\ta}%  Money Market
%        |%
%        c%                                    Stocks 
%        |%
%        c%                                    Bonds
%        |%
%        c%                                    Diversified
%        |%
%        >{\centering\hspace{0pt}}m{\the\tb}%  Stocks and Bonds
%        |%
%        >{\centering\hspace{0pt}}m{\the\tc}%  Money Market and Stocks
%        |%
%        c%                                    Others
%        |%
%      }
%      \hline
%      IMP&
%        Money Market&
%        Stocks&
%        Bonds&
%        Diversified&
%        Stocks and Bonds&
%        Money Market and Stocks&
%        Others\tabularnewline
%      \hline
%      1& 14.19\%& 57.71\%& 12.21\%& 4.50\%& 7.36\%& 3.04\%& 0.99\%\tabularnewline \hline
%      2& 14.08\%& 58.18\%& 12.32\%& 4.44\%& 7.30\%& 2.80\%& 0.88\%\tabularnewline \hline
%      3 &14.26\%& 58.09\%& 12.27\%& 4.50\%& 7.19\%& 2.75\%& 0.94\%\tabularnewline \hline
%      4 &13.94\%& 58.11\%& 12.14\%& 4.78\%& 7.35\%& 2.68\%& 0.99\%\tabularnewline \hline
%      5 &13.92\%& 58.13\%& 11.93\%& 4.56\%& 7.60\%& 2.98\%& 0.88\%\tabularnewline \hline
%    \end{tabular}
%  \end{center}
%  This table presents the allocations of the wealth in the IRA
%  and Keogh accounts in various asset classes.
%  Results from each set of imputed data are presented here.
%  The first column lists the number of the imputations,
%  and rest of the columns lists various allocations.
%  Entrees under each asset class show the percentage of investors
%  who have most of their IRA
%  and Keogh wealth invested in that particular asset class.
%  The asset class Diversified
%  includes stocks,
%  bonds,
%  and money market investments.
%  The asset class Others
%  include investments in various life insurance products,
%  annuities,
%  real estate, etc.
%  \medskip
%  \footnotesize SOURCE: Survey of Consumer Finances,
%  2001,
%  Federal Reserve Board,
%  USA.\par
%\end{table}
%  }
