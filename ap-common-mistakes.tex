\ProvidesFile{ap-common-mistakes.tex}[2024-05-05 common mistakes appendix]

\begin{VerbatimOut}{z.out}
\chapter{COMMON MISTAKES}

The following Headings, Mathematics, and Text
sections describe some common mistakes.




\section{Headings}

\textcite[page~289]{farkas2011}
wrote

\begin{quotation}
  The practice of stacking headings
  is routinely condemned by style manuals
  and other authorities.
  Here is a typical statement,
  taken from Houghton Mifflin's guidelines for authors.
  \begin{quotation}
    Avoid ``stacking'' heads,
    or placing two levels
    of headings together without intervening text.
    A heading cannot substitute
    for the transitional
    or introductory paragraphs
    that guide the reader through a chapter.
    Remember too that a chapter opening looks better in type
    when one
    or more paragraphs
    of text precede the first heading.
  \end{quotation}
\end{quotation}
\end{VerbatimOut}

\MyIO


\begin{VerbatimOut}{z.out}


\section{Mathematics}

\subsection{Put a little extra horizontal space before dx}
\end{VerbatimOut}
\MyIO


\begin{VerbatimOut}{z.out}


\section{Text}
\end{VerbatimOut}

\MyIO


\begin{VerbatimOut}{z.out}

\subsection{e.g.,}
\ix{e.g.}

``e.g.'' should always be followed by a comma.
\end{VerbatimOut}

\MyIO


\begin{VerbatimOut}{z.out}

\subsection{``et al.'' is an abbreviation}
\ix{et al.}

The phrase ``et al.''
is an abbreviation
and should always be followed by a period.
It should be in the normal font for your document---%
do not italicize or underline it.

Example:\\[6pt]
\indent\indent
\begin{tabular}{@{}ll@{}}
  input&   \verb+Thun et al.~used data from Santa Claus.+\\
  output&  Thun et al.~used data from Santa Claus.\\
  comment& my recommendation\\[6pt]
  input&   \verb+Thun et al. used data from Santa Claus.+\\
  output&  Thun et al. used data from Santa Claus.\\
  comment& too much space after period---\LaTeX\ thinks period is end of sentence\\[6pt]
  input&   \verb+Thun et al\@. used data from Santa Claus.+\\
  output&  Thun et al\@. used data from Santa Claus.\\
  comment& spacing is right but the ``et al.'' could occur at end of a line\\
\end{tabular}
\end{VerbatimOut}

\MyIO


\begin{VerbatimOut}{z.out}

\subsection{i.e.,}
\ix{i.e.}

``i.e.'' should always be followed by a comma.
\end{VerbatimOut}

\MyIO


\begin{VerbatimOut}{z.out}

\subsection{ldots}

Use ``1, 2, \ldots, 10''
instead of 1, 2, ..., 10''
\end{VerbatimOut}

\MyIO


\begin{VerbatimOut}{z.out}

\subsection{text math subscripts}

If you are using an English word
as a math subscript or subsubscript
typeset it in a roman font like this
`\(x_\text{max}\)'
instead of
`\(x_{max}\)'.
\end{VerbatimOut}

\MyIO


\begin{VerbatimOut}{z.out}

\subsection{ties}

Change the space in
`Dr. ', `Fig. ', `Mr. ', `Mrs. ', `Mx. ', `Prof. ', etc.,
to `\(\sim\)' so the period will only be followed by one
space and, for example, `Dr.' and `Smith' will be tied
together so they won't be split over two lines.
\end{VerbatimOut}

\MyIO
