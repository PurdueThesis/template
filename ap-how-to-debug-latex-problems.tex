\ProvidesFile{ap-how-to-debug-latex-problems}[2023-08-30 How to Debug LaTeX Problems]

\begin{VerbatimOut}{z.out}
\chapter
  [HOW TO DEBUG LATEX PROBLEMS]%
  {HOW TO DEBUG \LaTeX\ PROBLEMS}
\ix{debugging \LaTeX//How to Debug \LaTeXLogo\ Problems appendix}
\end{VerbatimOut}

\MyIO


\begin{VerbatimOut}{z.out}

Add more content here.
\todoerror{add more content here}
\end{VerbatimOut}

\MyIO


\ProvidesFile{ap-debugging.tex}[2023-06-16 debugging appendix]


In \LuaLaTeXLogo\ error messages,
the |l| in,
for example,
|l.987| means |line|.
|l.987| means |line 987|.

|! Undefined control sequence.|
|l.|\emph{line\_number} |\|\emph{command}|}|

The
\emph{command}
on
\emph{line\_number}
of the current file does not exist.
It may be misspelled.

\endinput

\begin{environment}
mispelled

too many }

use \begin{tabular} instead of \halign

\alpha
not in m'th mode

\show \var

\showthe \dimension


compile input often

providing ``too many'' causes extra arguementl to be printed

Don't use the |unravel| package for debugging.
It can cause more problems than it solves.

Use minimum working examples.


\begin{VerbatimOut}{z.out}
\chapter{DEBUGGING}
\ix{debugging//Debugging appendix}

\MyIO

Type 'R' to run LaTeX without showing errors.

\linenumbers
https://www.ametsoc.org/index.cfm/ams/publications/author-information/latex-author-info/preparing-a-latex-manuscript-for-submission/

Remove all files ending with |.aux|.
Those are temporary files
and wil be remade automatically by LaTeX.


\includeonly
\endinput   (can only do at level 0)
comment out with %

\begin{comment} \ldots\ \end{comment}
https://ctan.math.illinois.edu/macros/latex/contrib/comment/writeup.pdf

\end{document}



when you aren't in a text math (\(...\) or $...$ but I recommend \(...\))




Just want to make sure you know about the following:
after you recompile your document click on the icon to the
immediate right to show any errors.

There aren't any errors that have to do with your question.

A float is a table or a figure.  \hbox is a horizontal box.
There are 72.27 point (pt) per inch.

Putting \, (backslash comma) in math mode gives a tiny
bit of horizontal space.  You may want to put that before
"erfc".

You're right.  There is a problem.  I'll look at it with
fresh eyes this weekend.

I made a copy of the document and am making small changes to
it but it fails to compile,  (I thought making a copy of
it would not affect you.)  I'm confused.




\left   \right
\[    \]
\(   \)
{   }
\begin{whatever}   \end{whatever}


special characters:
# $ % & ~ _ ^ \ { }


don't use &#<digit><digit><digit><digut>& to try and get Unicode fonts in LaTeX

PurdueThesis does not support the inputenc package.  Use, for example, $beta$
for beta.
\let\savedbeta=\beta
beta = beta or $\savedbeta$

% in bibliography files


\qrmtz
can't find control sequence
it's either misspelled? ir your file or you didn't include a package that defined it

use formatting that will help you
  indenting

use tools that will help you


use Google

http://crab.rutgers.edu/~karel/latex/class4/class4.html
http://crab.rutgers.edu/~karel/latex/class5/class5.html
http://crab.rutgers.edu/~karel/latex/class6/class6.html





Michelle Krummel has a good Errors and Debugging tutorial.
See LaTeX Tutorial 7 below.  In case you're interested in
her other LaTeX tutorials they are in the table also.

NAME   DATE   LENGTH
LaTeX Tutorial 1 - Creating a LaTeX Document&
LaTeX Tutorial 2 - Common Mathematical Notation&
LaTeX Tutorial 3 - Brackets, Tables, and Arrays&
LaTeX Tutorial 4 - Creating Lists&
LaTeX Tutorial 5 - Text and Document Formatting&
LaTeX Tutorial 6: Packages, Macros and Graphics&
LaTeX Tutorial 7: Errors and Debugging&
LaTeX Tutorial 8: TeXmaker and Overleaf Tips&
LaTeX Tutorial 9 - Calculus Notation&
LaTeX Tutorial 10: How to Format a Math Paper&
LaTeX Tutorial 11: Beamer Slide Presentation&

@online{krummel?,
  author  = {Michelle Krummel},
  date    = {2020-07-08},
  note    = {35 minutes long},
  title   = {LaTeX Tutorial 1 - Creating a LaTeX Document},
  url     = {https://www.youtube.com/watch?v=0ivLZh9xK1Q},
  urldate = {2022-04-15},
}

@online{krummel?,
  author  = {Michelle Krummel},
  date    = {2020-07-27},
  note    = {36 minutes long},
  title   = {LaTeX Tutorial 2 - Common Mathematical Notation},
  url     = {https://www.youtube.com/watch?v=bCumVPGR4ts},
  urldate = {2022-04-15},
}

@online{krummel?,
  author  = {Michelle Krummel},
  date    = {2020-07-27},
  note    = {42 minutes long},
  title   = {LaTeX Tutorial 3 - Brackets, Tables, and Arrays},
  url     = {https://www.youtube.com/watch?v=kefvRACdXHs},
  urldate = {2022-04-15},
}

@online{krummel?,
  author  = {Michelle Krummel},
  date    = {2020-07-27},
  note    = {12 minutes long},
  title   = {LaTeX Tutorial 4 - Creating Lists},
  url     = {https://www.youtube.com/watch?v=dZitO3IJTys},
  urldate = {2022-04-15},
}

@online{krummel?,
  author  = {Michelle Krummel},
  date    = {2020-07-27},
  note    = {23 minutes long},
  title   = {LaTeX Tutorial 5 - Text and Document Formatting},
  url     = {https://www.youtube.com/watch?v=3KvsemMjHPU},
  urldate = {2022-04-15},
}

@online{krummel?,
  author  = {Michelle Krummel},
  date    = {2020-08-23},
  note    = {30 minutes},
  title   = {LaTeX Tutorial 6: Packages, Macros and Graphics},
  url     = {https://www.youtube.com/watch?v=L7WzLrzU2Ec},
  urldate = {2022-04-15},
}

@online{krummel?,
  author  = {Michelle Krummel
  date    = {2020-08-25},
  note    = {23 minutes long},
  title   = {LaTeX Tutorial 7: Errors and Debugging},
  url     = {https://www.youtube.com/watch?v=bYoTJc81-qk},
  urldate = {2022-04-15},
}
pp
@online{krummel?,
  author  = {Michelle Krummel},
  date    = {2020-08-25},
  note    = {41 minutes long},
  title   = {LaTeX Tutorial 8: TeXmaker and Overleaf Tips},
  url     = {https://www.youtube.com/watch?v=DzwoiIxrihA},
  urldate = {2022-04-15},
}

@online{krummel?,
  author  = {Michelle Krummel},
  date    = {2015-05-03},
  note    = {29 minutes long},
  title   = {LaTeX Tutorial 9 - Calculus Notation},
  url     = {https://www.youtube.com/watch?v=eDeRsE-NTb4},
  urldate = {2022-04-15},
}

@online{krummel?,
  author  = {Michelle Krummel},
  date    = {2016-10-03},
  note    = {25 minutes long},
  title   = {LaTeX Tutorial 10: How to Format a Math Paper},
  url     = {https://www.youtube.com/watch?v=FcVP3gGUtGI},
  urldate = {2022-04-15},
}

@online{krummel?,
  author  = {Michelle Krummel,
  date    = {2017-02-17},
  note    = {47 minutes long},
  tittle  = {LaTeX Tutorial 11: Beamer Slide Presentation},
  url     = {https://www.youtube.com/watch?v=0fsWGg81RwU},
  urldate = {2022-04-15},
}

I like powerdot better
Hendri Adriaens
The powerdot class
https://ctan.org/pkg/powerdot
2021/05/19
version = {1.7},
urldate = {2022-04-15}
